
% Szkielet dla pracy inżynierskiej pisanej w języku polskim.

\documentclass[polish,master,a4paper,oneside]{ppfcmthesis}


\usepackage[utf8]{inputenc}
\usepackage[OT4]{fontenc}
\usepackage{graphicx}
\usepackage{float}


% Authors of the thesis here. Separate them with \and
\author{%
   Wojciech Kopeć \album{101675} \and
} 
\title{Imitation Learning}                   % Note how we protect the final title phrase from breaking
\ppsupervisor{dr~inż.~Krzysztof Dembczyński} % Your supervisor comes here.
\ppyear{2017}                                         % Year of final submission (not graduation!)


\begin{document}

% Front matter starts here
\frontmatter\pagestyle{empty}%
\maketitle\cleardoublepage%

% Blank info page for "karta dyplomowa"
\thispagestyle{empty}\vspace*{\fill}%
\begin{center}Tutaj przychodzi karta pracy dyplomowej;\\oryginał wstawiamy do wersji dla archiwum PP, w pozostałych kopiach wstawiamy ksero.\end{center}%
\vfill\cleardoublepage%

% Table of contents.
\pagenumbering{Roman}\pagestyle{ppfcmthesis}%
\tableofcontents* \cleardoublepage%

% Main content of your thesis starts here.
\mainmatter%
W poniższym rozdziale przedstawione są badane i zaimplementowane w pracy podejścia.

Każde z rozwiązań działa w ramach wspólnego szkieletu, bazującego na przykładowych rozwiązaniach towarzyszących środowisku VizDoom. Dzięki temu możliwe jest bezpośrednie porównanie zachowania różnych podejść przy zmianie tylko kluczowych algorytmów przy zachowaniu niezmienności pozostałych czynników.

\chapter{Eksperymenty}


% All appendices and extra material, if you have any.
\cleardoublepage\appendix%
%\input{0a-zalacznik.tex}
%\input{0b-pisanie-w-latexu.tex}

% Bibliography (books, articles) starts here.
\bibliographystyle{plalpha}{\raggedright\sloppy\small\bibliography{bibliography}}

% Colophon is a place where you should let others know about copyrights etc.
\ppcolophon

\end{document}
