\chapter{Uczenie na podstawie informacji obrazowej}

W poniższym rozdziale opisano podejście zastosowane w pracy \cite{mnih2015human}, które pozwoliło na skuteczną naukę na podstawie surowych danych obrazowych. Dalej przedstawiono wybrane techniki zwiększające skuteczność praktycznego uczenia ze wzmocnieniem. Na końcu opisano środowisko 3D VizDoom, oraz scenariusze na których przeprowadzono eksperymenty obliczeniowe.
\section{Aproksymatory i głębokie sieci Q}

Zdefiniowana w podrozdziale \ref{methods} przykładowa reguła aktualizacji wartości funkcji Q wygląda następująco:

$$Q(s,a) \leftarrow Q(s,a) + \alpha (R(s,s') + \gamma (max_{a'}Q(s',a') - Q (s,a)))$$

Wynika z niej, że po wykonaniu ruchu wartość funkcji Q dla poprzedniego stanu aktualizujemy na podstawie otrzymanej nagrody i wartości funkcji Q dla stanu aktualnego. Oznacza to, że dla każdego stanu, który analizujemy, konieczna jest znajomość jego wartości funkcji Q dla wszystkich możliwych akcji. Oznacza to, że dla dokładnego przedstawienia funkcji $Q(s,a)$  konieczne jest zapamiętanie $\left\vert{S}\right\vert \cdot \left\vert{A}\right\vert$ wartości. Co więcej, aby uzyskać sensowne wartości tej funkcji konieczne jest odwiedzenie każdego ze stanów wiele razy, zanim aktualizowana stopniowo wartość funkcji Q będzie bliska prawdziwej. Wiele z tych stanów jest też bardzo podobnych do siebie nawzajem, więc wiedza wyniesiona dla jednego stanu powinna się w pewien sposób generalizować na podobne stany.

Backgammon, jedna z gier planszowych służących jako benchmark algorytmów uczenia ze wzmocnieniem, ma $10^{20}$ możliwych stanów, a szachy $10^{40}$. Jeden obraz 90x60 pixeli w skali szarości, używany jako zapis stanu w problemie rozwiązywanym w ramach poniższej pracy może przyjąć $256^{5400}$ różnych kombinacji. Wiele realnych problemów opisanych jest wartościami ciągłymi, nie dyskretnymi, a praktyczna liczba ich możliwych stanów rośnie wykładniczo wraz ze wzrostem dokładności pomiaru.

\subsection{Aproksymatory funkcji Q}

Rozważanie i zapamiętanie każdego stanu z osobna dla bardziej skomplikowanych problemów jest niemożliwe i niepraktyczne ze względu na liczbę możliwych stanów i podobieństwo wielu stanów. Rozwiązaniem jest wykorzystanie \textit {aproksymatora funkcji Q} - niestablicowanej, parametrycznej funkcji pary (stan,akcja) $\hat{Q}_{\theta}(s,a)$, gdzie $\theta$ jest wektorem parametrów funkcji.


Aproksymator (za \cite{wjaskowski2016}):
\begin{itemize}
\item musi być łatwo obliczalny,
\item kompresuje dużą przestrzeń stanów w znacznie mniejszą przestrzeń parametrów,
\item uogólnia wiedzę na temat podobnych stanów.
\item w większości przypadków przyspiesza uczenie w stosunku do wersji stablicowanej ze względu na uogólnianie wiedzy
\end{itemize}

Jako jedne z pierwszych i prostszych aproksymatorów stosowano funkcje liniowe, opierające się na ręcznie zdefiniowanych cechach: $\hat{Q}_{\theta}(s,a) = \theta_0 + \theta_1x_1 + \theta_2x_2 + ... + \theta_nx_n$, gdzie wektor $x = (x_1, x_2, …, x_n)$ jest wektorem cech. Przykładem zastosowania może być gra w warcaby, opisana w \cite{Samuel:1959:SML:1661923.1661924}. Zaletami liniowego aproksymatora są prostota i łatwość interpretacji, a także szybkość obliczania i nauki. 
Dalszym krokiem było wykorzystanie sieci neuronowych jako aproksymatorów w grze w Backgammona \cite{Tesauro1992451}. W pierszej wersji algorytmu wykorzystano ręcznie zaprojektowane cechy, w kolejnych wykorzystano prawie surową informację o rozkładzie pionków na planszy. Sieć neuronowa jest bardziej skomplikowanym i trudniejszym do nauczenia aproksymatorem, ale jest w stanie zamodelować znacznie bardziej złożone funkcje.

\subsection{Uczenie na podstawie surowych danych obrazowych - Atari 2600}

Jednym z największych przełomów uczenia ze wzmocnieniem ostatnich lat była praca \cite{mnih2015human}, w której autorzy wykorzystali głębokie sieci neuronowe do stworzenia agenta potrafiącego grać na ludzkim poziomie w klasyczne gry z Atari 2600, wykorzystując jako reprezentację stanu jedynie surowy zapis obrazu 2D. Dotychczas, jak w poprzednich przykładach, algorytmy uczenia ze wzmocnieniem opierały się na manualnie stworzonej reprezentacji stanów. W \cite{mnih2015human} pokazano, że możliwe jest stworzenie rozwiązania, które samo będzie potrafiło ekstrahować wysokopoziomowe cechy z niskopoziomowych danych. Zaproponowana architektura, jak również pomysłowe usprawnienia zwiększające stabilność uczenia zaproponowane w artykule, a opisane w rozdziale \ref {enhancements} stanowią obecnie podstawę i punkt odniesienia dla większości dalszych badań na temat uczenia ze wzmocnieniem.

Jako aproksymator funkcji Q wykorzystano głęboką sięć neuronową. Z tego powodu opisywane podejście określa się często skrótem DQN \textit{(ang. Deep Q Network)}, czyli głęboka sieć Q. Analogicznie jak w obecnie stosowanych architekturach rozpoznawania obrazu, pierwsze warstwy sieci to warstwy konwolucyjne, które wykrywają kolejno nisko i wysokopoziomowe cechy obrazu. Dalsze warstwy, w pełni połączone, łączą informacje z warstw konwolucyjnych we wnioski na temat stanu świata, na podstawie których następne warstwy mogą określić wartość funkcji Q.



\section{Q-learning - usprawnienia}\label{enhancements}
Skuteczność i stabilność Q-learningu może zostać drastycznie polepszona dzięki zastosowaniu następujących technik.

\subsection{Pamięć powtórek}

Szkielet uczenia ze wzmocnieniem opiera się na zbieraniu doświadczeń i uaktualnianiu na ich podstawie stanu wiedzy agenta. W praktyce, doświadczenia zbierane bezpośrenio po sobie są silnie skorelowane - przykładowo agent uczący się na podstawie obrazu jazdy samochodem w kolejnych klatkach widzi niemal identyczne obrazy i wykonuje najczęsciej te same akcje. Oznacza to, że aktualizowanie wiedzy agenta na podstawie nowych doświadczeń, czy to pojedynczo czy w paczkach, będzie skutkować funkcją obciążoną w kierunku tych, nowych doświadczeń.

Aby temu zapobiec w [REF] zaproponowano metodę pamięci powtórek \textit{(ang. Replay memory)}. Metoda opiera się na zapamiętywaniu znacznej ilości najnowszych doświadczeń. Po każdym kroku nowe doświadczenia dodawane są do pamięci (w przypadku braku miejsca zastępując nastarsze), a następnie z pamięci wybierana jest losowa próbka doświadczeń, na podstawie których aktualizowana jest wiedza agenta. Dzieki tej technice dane użyte do nauki przez agenta są nieskorelowane i niezależne. Dodatkowo, dzięki dostępowi do starszych danych agent jest mniej podatny na obniżanie jakości gry na skutek krótkotrwałych spadków wyników.

Dalsze rozszerzenia metody mają na celu np. priorytetyzowanie używania do nauki najważniejszych doświadczeń [REF].
\subsection{Zamrażanie docelowej sieci}

Podobnie jak Pamięć powtórek, zamrażanie docelowej sieci \textit{(ang. Target network freezeing / fixed target network)} służy zmniejszeniu skutków obciążenia rozkładu danych uczących zebranych przez agenta, a wynikającego ze sposobu zbierania próbek. Zamrażanie sieci zakłada utrzymywanie dwóch funkcji Q - starej i nowej. Agent działa na podstawie nowej funkcji, ale wartości Q ,,docelowych'' stanów używanych do aktualizacji wartości Q (\ref{tdl}) pobierane są ze starej funkcji. Co jakiś czas do starej funkcji Q przepisywana jest nowa funkcja Q.

Technika ma na celu zniwelowanie oscylacji i ustabilizowanie zachowań agenta. Dzięki wykorzystaniu ,,zamrożonych'' wartości do nauki funkcji Q zerwane jest sprzężenie zwrotne pomiędzy zebranymi danymi a wartościami docelowymi.

\subsection{Kształtowanie}

W wielu zadaniach stawianych przed uczeniem ze wzmocnieniem osiągnięcie celu jest bardzo trudne, a agent dostaje nagrody dopiero po osiągnięciu stanów terminalnych, albo na zaawansowanym etapie zadania. Agent uczący się na podstawie prób, błędów i losowych akcji nie jest najczęściej w stanie wykonać wystarczająco dużej części zadania, żeby dostać informację zwrotną w postaci nagrody, a więc nie ma jak się uczyć lub uczenie następuje bardzo wolno.

Kształtowanie \textit{(ang. Shaping)} zakłada sztuczne wprowadzenie do środowiska dodatkowych nagród, które agent będzie dostawał po wykonaniu etapów pośrednich zadania. Przykładowo, przy grze w szachy, w której agent dostaje nagrodę tylko za wygraną lub przegraną (1 lub -1) można byłoby wprowadzić nagrodę 0.1 za zbijanie figur przeciwnika.

Kształtowanie wymaga możliwości ingerencji w środowisko albo percepcję agenta (rozpoznawanie, kiedy agent powinien dostać sztuczną nagrodę i ingerowanie w odczyty nagrody dokonywane przez agenta). Co ważniejsze, wymaga wiedzy eksperckiej na temat zadania wykonywanego przez agenta (możliwość określenia sensownych etapów zadania, na których agent miałby dostać sztuczną nagrodę) i wiedzy na temat środowiska, w którym agent się porusza (wysokość sztucznej nagrody musi być dopasowana do prawdziwych nagród, które może dostawać agent). Dodatkowo, kroki określone przez eksperta mogą wymuszać nieoptymalną politykę działania i powstrzymać agenta przed odkryciem optymalnych strategii.

\section{Środowisko VizDoom}

Środowisko VizDoom, przedstawione w \cite{DBLP:journals/corr/KempkaWRTJ16}, jest narzędziem do testowania algorytmów sterowania na podstawie surowych danych o obrazie 3D. Środowisko bazujące na klasycznej grze Doom, w której gracz widzi świat z perspektywy pierwszej osoby i strzela do piekielnych potworów.

\section{Scenariusze}
Eksperymenty przeprowadzono na następujących scenariuszach.

\subsection{Basic}
Sceneria składa się z prostokątnego pomieszczenia. Agent jest w jednym końcu pomieszczenia, a w losowym miejscu pod przeciwległą ścianą jest pojedynczy, nieruchomy przeciwnik. Agent może atakować i poruszać się bokiem w lewo i prawo. Strategia optymalna polega na przesunięciu się w kierunku przeciwnika i oddaniu pojedynczego strzału.
\subsection{Defend the center}
Sceneria składa się z kolistej areny. Agent jest na środku areny, a na jej krańcach losowo pojawiają się przeciwnicy, którzy poruszają się w stronę agenta, a po dotarciu do niego atakują. Agent może atakować i kręcić się w okół własnej osi w lewo i prawo. Strategia optymalna polega na kręceniu się w kółko, ignorowaniu odległych przeciwników i strzelaniu do biskich.

