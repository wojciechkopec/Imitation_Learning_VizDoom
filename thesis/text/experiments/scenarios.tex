\section{Scenariusze}
Eksperymenty przeprowadzono na następujących scenariuszach.

\subsection{Basic}
Sceneria składa się z prostokątnego pomieszczenia. Agent jest w jednym końcu pomieszczenia, a w losowym miejscu pod przeciwległą ścianą jest pojedynczy, nieruchomy przeciwnik. Agent może atakować i poruszać się bokiem w lewo i prawo. Strategia optymalna polega na przesunięciu się w kierunku przeciwnika i oddaniu pojedynczego strzału.
\subsection{Defend the center}
Sceneria składa się z kolistej areny. Agent jest na środku areny, a na jej krańcach losowo pojawiają się przeciwnicy, którzy poruszają się w stronę agenta, a po dotarciu do niego atakują. Agent może atakować i kręcić się w okół własnej osi w lewo i prawo. Strategia optymalna polega na kręceniu się w kółko, ignorowaniu odległych przeciwników i strzelaniu do biskich.
