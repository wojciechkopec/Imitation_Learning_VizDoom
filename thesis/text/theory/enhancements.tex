\section{Q-learning - usprawnienia}\label{enhancements}
Skuteczność i stabilność Q-learningu może zostać drastycznie polepszona dzięki zastosowaniu następujących technik.

\subsection{Pamięć powtórek}{replaymemory}

Szkielet uczenia ze wzmocnieniem opiera się na zbieraniu doświadczeń i uaktualnianiu na ich podstawie stanu wiedzy agenta. W praktyce, doświadczenia zbierane bezpośrenio po sobie są silnie skorelowane - przykładowo agent uczący się na podstawie obrazu jazdy samochodem w kolejnych klatkach widzi niemal identyczne obrazy i wykonuje najczęsciej te same akcje. Oznacza to, że aktualizowanie wiedzy agenta na podstawie nowych doświadczeń, czy to pojedynczo czy w paczkach, będzie skutkować funkcją obciążoną w kierunku tych, nowych doświadczeń.

Aby temu zapobiec w \cite{mnih2015human} zaproponowano metodę pamięci powtórek \textit{(ang. Replay memory)}. Metoda opiera się na zapamiętywaniu znacznej ilości najnowszych doświadczeń. Po każdym kroku nowe doświadczenia dodawane są do pamięci (w przypadku braku miejsca zastępując nastarsze), a następnie z pamięci wybierana jest losowa próbka doświadczeń, na podstawie których aktualizowana jest wiedza agenta. Dzieki tej technice dane użyte do nauki przez agenta są nieskorelowane i niezależne. Dodatkowo, dzięki dostępowi do starszych danych agent jest mniej podatny na obniżanie jakości gry na skutek krótkotrwałych spadków wyników.

Dalsze rozszerzenia metody mają na celu np. priorytetyzowanie używania do nauki najważniejszych doświadczeń \ref{DBLP:journals/corr/SchaulQAS15}.
\subsection{Zamrażanie docelowej sieci}\label{fixedtarget}

Podobnie jak Pamięć powtórek, zamrażanie docelowej sieci \textit{(ang. Target network freezeing / fixed target network)} zostało zaproponowane w \cite{mnih2015human} i służy zmniejszeniu skutków obciążenia rozkładu danych uczących zebranych przez agenta, a wynikającego ze sposobu zbierania próbek. Zamrażanie sieci zakłada utrzymywanie dwóch funkcji Q - starej i nowej. Agent działa na podstawie nowej funkcji, ale wartości Q ,,docelowych'' stanów używanych do aktualizacji wartości Q (\ref{tdl}) pobierane są ze starej funkcji. Co jakiś czas do starej funkcji Q przepisywana jest nowa funkcja Q.

Technika ma na celu zniwelowanie oscylacji i ustabilizowanie zachowań agenta. Dzięki wykorzystaniu ,,zamrożonych'' wartości do nauki funkcji Q zerwane jest sprzężenie zwrotne pomiędzy zebranymi danymi a wartościami docelowymi.

\subsection{Kształtowanie}\label{shaping}
W wielu zadaniach stawianych przed uczeniem ze wzmocnieniem osiągnięcie celu jest bardzo trudne, a agent dostaje nagrody dopiero po osiągnięciu stanów terminalnych, albo na zaawansowanym etapie zadania. Agent uczący się na podstawie prób, błędów i losowych akcji nie jest najczęściej w stanie wykonać wystarczająco dużej części zadania, żeby dostać informację zwrotną w postaci nagrody, a więc nie ma jak się uczyć lub uczenie następuje bardzo wolno.

Kształtowanie \textit{(ang. Shaping)} (\cite{Mataric94rewardfunctions}) zakłada sztuczne wprowadzenie do środowiska dodatkowych nagród, które agent będzie dostawał po wykonaniu etapów pośrednich zadania. Przykładowo, przy grze w szachy, w której agent dostaje nagrodę tylko za wygraną lub przegraną (1 lub -1) można byłoby wprowadzić nagrodę 0.1 za zbijanie figur przeciwnika.

Kształtowanie wymaga możliwości ingerencji w środowisko albo percepcję agenta (rozpoznawanie, kiedy agent powinien dostać sztuczną nagrodę i ingerowanie w odczyty nagrody dokonywane przez agenta). Co ważniejsze, wymaga wiedzy eksperckiej na temat zadania wykonywanego przez agenta (możliwość określenia sensownych etapów zadania, na których agent miałby dostać sztuczną nagrodę) i wiedzy na temat środowiska, w którym agent się porusza (wysokość sztucznej nagrody musi być dopasowana do prawdziwych nagród, które może dostawać agent). Dodatkowo, kroki określone przez eksperta mogą wymuszać nieoptymalną politykę działania i powstrzymać agenta przed odkryciem optymalnych strategii. 29 - 1994,66,86
