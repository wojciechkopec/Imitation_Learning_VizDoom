\section{Metody}\label{methods}

Podejścia stosowane do uczenia ze wzmocnieniem możemy podzielić na trzy rodzaje, w zależności od typu informacji na której bazuje agent \cite{Russell:2009:AIM:1671238}.

\begin{enumerate}
\item Agent z polityką - uczy się polityki  $\pi: S \rightarrow A$. Przykłady:
\begin{itemize}
\item algorytmy ewolucyjne,
\item uczenie przez demonstrację.
\end{itemize}
\item Agent z funkcją użyteczności $U$. Przykłady:
\begin{itemize}
\item adaptatywne programowanie dynamiczne \textit{(ang. adaptative dynamic programming, ADP)},
\item metoda różnic czasowych \textit{(ang. temporal difference learning, TDL)}.
\end{itemize}
\item Agent z funkcją użyteczności $Q$. Przykłady:
\begin{itemize}
\item Q-learning,
\item SARSA \textit{(ang. State, Action, Reward, State, Action)}
\end{itemize}
\end{enumerate}

Poniżej przedstawione zostaną najważniejsze metody.

\subsection{Metoda różnic czasowych}\label{tdl}

Metoda różnic czasowych \textit{(ang. Temporal difference learning, TDL)} \cite{Sutton:1988:LPM:637912.637937} opiera się na uaktualnianiu stanu wiedzy agenta na podstawie różnicy pomiędzy spodziewanym a zaobserwowanym wynikiem.

Agent trafiający do stanu $s'$ po wykonaniu akcji $a$ w stanie $s$ może uaktualnić stan swojej wiedzy: $U(s) \leftarrow U(s) + \alpha (R(s,s') + \gamma U(s') - U (s))$ lub $Q(s,a) \leftarrow Q(s,a) + \alpha (R(s,s') + \gamma (\max_{a'}Q(s',a') - Q (s,a))),$ gdzie $\alpha$ jest współczynnikiem prędkości uczenia. Jeżeli $\alpha$ w odpowiedni sposób zmniejsza się w czasie, to TDL gwarantuje zbieżność do optimum globalnego.

\subsection{Funkcja U, Q i SARSA}\label{qlearning}
\begin{itemize}
\item Funkcja U (patrz: rozdział \ref{mdp}) opisuje użyteczność stanu,
\item Funkcja Q (patrz: rozdział \ref{mdp}) opisuje użyteczność wykonania danej akcji w danym stanie,
\item SARSA stanowi wariację metody Q. W Q-learningu wartość funkcji Q jest aktualizowana na podstawie wartości Q dla najlepszej akcji do wykonywania w stanie $s'$ ($\gamma \max_{a'}Q(s',a') - Q (s,a))$), natomiast w SARSIE na podstawie wykonanej przez agenta akcji ($\gamma (Q(s',a') - Q (s,a)))$), czyli przebytej przez agenta trajektorii $ s \rightarrow a \rightarrow s' \rightarrow a'$. Aktualizacja TD w SARSA-ie wygląda następująco:

$$Q(s,a) \leftarrow Q(s,a) + \alpha (R(s,s') + \gamma (Q(s',a') - Q (s,a)))$$

SARSA może dla niektórych problemów zachowywać się nieznacznie lepiej niż Q-learning, ale w większości przypadków będzie się uczyła wolniej bez wpływu na jakość agenta.
\end{itemize}

Mimo podobnych wzorów i definicji nauka funkcji Q ma jedną, diametralną przewagę na nauką funkcji U - funkcja Q nie wymaga znajomości modelu świata do wyboru najlepszej akcji do wykonania. Zbiór dostępnych akcji $A$ jest znany agentowi. Przy wyborze najlepszej akcji $a$ w stanie $s$:
\begin{itemize}
\item Agent z funkcją Q wybiera akcję $a = argmax_{a \in A} Q(s,a)$.

\item Agent z funkcją U wybiera akcję, która maksymalizuje $U(s')$ - wartość stanu, do którego trafi agent: $a = argmax_{a \in A} \sum_{s'} T_a(s,s')U(s')$. Obliczenie tego wyrażenia wymaga znajomości modelu przejść $T_a(\cdot, \cdot)$, czyli modelu świata. Można przyjąć, że dla trudniejszych i realnych problemów model świata nie jest dostępny.
\end{itemize}

Z tego powodu większość wiodących rozwiązań w dziedzinie uczenia ze wzmocnieniem oparta jest na Q-learningu.  Dalsza część pracy przyjmuje Q-learning jako obowiązującą metodę rozwiązywania problemu uczenia ze wzmocnieniem.

Niezależnie od metody, dla realnych i interesujących naukowo problemów uczenie ze wzmocnieniem jest wymagające obliczeniowo i czasowo. Mimo wspomagających agenta technik, nauka sprowadza się najczęściej do interakcji ze środowiskiem metodą prób i błędów - potrzeba wiele prób i błędów, zanim agent zacznie pojmować zasady rządzące środowiskiem w którym się znajduje, a potem dużo dalszych zanim znajdzie dla danego środowiska satysfakcjonującą skuteczną politykę działania.

