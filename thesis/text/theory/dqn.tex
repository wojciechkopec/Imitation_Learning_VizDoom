\section{Aproksymatory i głębokie sieci Q}

Zdefiniowana w podrozdziale \ref{methods} przykłądowa reguła aktualizacji wartości funkcji Q wygląda następująco:

$$Q(s,a) \leftarrow Q(s,a) + \alpha (R(s,s') + \gamma max_{a'}Q(s',a') - Q (s,a))$$

Wynika z niej, że po wykonaniu ruchu wartość funkcji Q dla poprzedniego stanu aktualizujemy na podstawie otrzymanej nagrody i wartości funkcji Q dla stanu aktualnego. Oznacza to, że dla każdego stanu, który analizujemy, konieczna jest znajomość jego wartości funkcji Q dla wszystkich możliwych akcji. Oznacza to, że dla dokładnego przedstawienia funkcji $Q(s,a)$  konieczne jest zapamiętanie $\left\vert{S}\right\vert \cdot \left\vert{A}\right\vert$ wartości. Co więcej, aby uzyskać sensowne wartości tej funkcji konieczne jest odwiedzenie każdego ze stanów wiele razy, zanim aktualizowana stopniowo wartość funkcji Q będzie bliska prawdziwej. Wiele z tych stanów jest też bardzo podobnych do siebie nawzajem, więc wiedza wyniesiona dla jednego stanu powinna się w pewien sposób generalizować na podobne stany.

Backgammon, jedna z gier planszowych służących jako benchmark algorytmów uczenia ze wzmocnieniem, ma $10^{20}$ możliwych stanów, a szachy $10^{40}$. Jeden obraz 90x60 pixeli w skali szarości, używany jako zapis stanu w problemie rozwiązywanym w ramach poniższej pracy może przyjąć $256^{5400}$ różnych kombinacji. Wiele realnych problemów opisanych jest wartościami ciągłymi, nie dyskretnymi, a praktyczna liczba ich możliwych stanów rośnie wykładniczo wraz ze wzrostem dokładności pomiaru.

\subsection{Aproksymatory funkcji Q}

Rozważanie i zapamiętanie każdego stanu z osobna dla bardziej skomplikowanych problemów jest niemożliwe i niepraktyczne ze względu na liczbę możliwych stanów i podobieństwo wielu stanów. Rozwiązaniem jest wykorzystanie \textit {aproksymatora funkcji Q} - niestablicowanej, parametrycznej funkcji pary (stan,akcja) $\hat{Q}_{\theta}(s,a)$, gdzie $\theta$ jest wektorem parametrów funkcji.


Aproksymator (za \cite{wjaskowski2016}):
\begin{itemize}
\item musi być łatwo obliczalny,
\item kompresuje dużą przestrzeń stanów w znacznie mniejszą przestrzeń parametrów,
\item uogólnia wiedzę na temat podobnych stanów.
\item w większości przypadków przyspiesza uczenie w stosunku do wersji stablicowanej ze względu na uogólnianie wiedzy
\end{itemize}

Jako jedne z pierwszych i prostszych aproksymatorów stosowano funkcje liniowe, opierające się na ręcznie zdefiniowanych cechach: $\hat{Q}_{\theta}(s,a) = \theta_0 + \theta_1cecha1 + \theta_2cecha2 + ... + \theta_ncechan$. Przykładem zastosowania może być gra w warcaby, opisana w \cite{Samuel:1959:SML:1661923.1661924}. Zaletami liniowego aproksymatora są prostota i łatwość interpretacji, a także szybkość obliczania i nauki. 
Dalszym krokiem było wykorzystanie sieci neuronowych jako aproksymatorów w grze w Backgammona \cite{Tesauro1992451}. W pierszej wersji algorytmu wykorzystano ręcznie zaprojektowane cechy, w kolejnych wykorzystano prawie surową informację o rozkłądzie pionków na planszy. Sieć neuronowa jest bardziej skomplikowanym i trudniejszym do nauczenia aproksymatorem, ale jest w stanie zamodelować znacznie bardziej złożone funkcje.

\subsection{Uczenie na podstawie surowych danych obrazowych - Atari 2600}

Jednym z największych przełomów uczenia ze wzmocnieniem ostatnich lat była praca \cite{mnih2015human}, w której autorzy wykorzystali głębokie sieci neuronowe do stworzenia agenta potrafiącego grać na ludzkim poziomie w klasyczne gry z Atari 2600, wykorzystując jako reprezentację stanu jedynie surowy zapis obrazu 2D. Dotychczas, jak w poprzednich przykładach, algorytmy uczenia ze wzmocnieniem opierały się na ekspercko wykreowanej reprezentacji stanów - \cite{mnih2015human} pokazali, że możliwe jest stworzenie rozwiązania, które samo będzie potrafiło ekstrahować wysokopoziomowe cechy z niskopoziomowych danych. Zaproponowana architektura, jak również pomysłowe usprawnienia zwiększające stabilność uczenia zaproponowane w artykule, a opisane w rozdziale \ref {enhancements} stanowią obecnie podstawę i punkt odniesienia dla większości dalszych badań na temat uczenia ze wzmocnieniem.

Jako aproksymator funkcji Q wykorzystano głęboką sięć neuronową. Z tego powodu opisywane podejście określa się często skrótem DQN \textit{ang. Deep Q Network)}, czyli głęboka sieć Q. Zastosowana architektura wygląda następujaco:

[RYSUNEK SIECI]

Analogicznie jak w obenie stosowanych architekturach rozpoznawania obrazu, pierwsze warstwy sieci to warstwy konwolucyjne, które wykrywają kolejno nisko i wysokopoziomowe cechy obrazu. Dalsze warstwy, w pełni połączone, łączą informacje z warstw konwolucyjnych we wnioski na temat stanu świata, na podstawie których następne warstwy mogą określić wartość funkcji Q.


