\section{Odwrócone uczenie ze wzmocnieniem}
Użycie kopiowania zachowań i wywodzących sie z niego metod pozwala na wytrenowanie agenta odruchowego[SPRAWDZIĆ], który dla danego stanu będzie potrafił określić optymalną akcję na jeden krok do przodu. Taki agent zna optymalną politykę, ale nie jest świadomy jej powodów. Odwrócone uczenie ze wzmocnieniem  \textit{(ang. Inverse reinforcement learning, IRL)} opiera się na założeniu, że w ogólności optymalna polityka agenta nie stanowi najlepszego i najbardziej zwięzłego opisu zadania podstawionego przed agentem - najbardziej precyzyjnym opisem, pozwalającym na większą dowolność i adaptację jest znajomość funkcji nagród $R_a(\cdot,\cdot)$.

Oczywiście, bez modelu środowiska funkcja $R_a(\cdot,\cdot)$ nie jest dostępna, dlatego zadaniem postawionym przed odwróconym uczeniem ze wzmocnieniem jest odtworzenie $R_a(\cdot,\cdot)$ na podstawie dostarczonych trajektorii eksperta.

Zadanie odwróconego uczenia ze wzmocnieniem jest znacznie trudniejsze od zwykłego uczenia ze wzmocnieniem. Przede wszystkim, IRL musi się zmierzyć z 2 problemami:

\begin{itemize}
\item Niejasność $R_a(\cdot,\cdot)$ - w większości przypadków, dla danych trajektorii eksperta istnieje nieskończenie wiele pasujących $R_a(\cdot,\cdot)$. Formalnie oznacza to, że IRL nie ma zdefiniowanego poprawnego rozwiązania.
\item Złożoność obliczeniowa - samo uczenie ze wzmocnieniem jest bardzo wymagające obliczeniowo. W odwróconym uczeniu ze wzmocnieniem, sprawdzanie $R_a(\cdot,\cdot)$ uzyskanych w kolejnych krokach wymaga każdorazowego rozwiązywania problemu uczenia ze wzmocnieniem na podstawie aktualnej funkcji $R_a(\cdot,\cdot)$, co oznacza, że IRL wymaga większego o rząd wielkości kosztu obliczeniowego niż RL.
\end{itemize}

Przykładowe zastosowania odwróconego uczenia ze wzmocnieniem to parkowanie samochodu \cite{DBLP:conf/iros/AbbeelDNT08}, nawigacja na podstawie obrazów satelitarnych \cite{Ratliff:2006:MMP:1143844.1143936} albo wykonywanie ewolucji helipkopterem\cite{DBLP:journals/cacm/CoatesAN09}.


