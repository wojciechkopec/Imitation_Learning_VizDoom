\section{Słownik pojęć}

W dalszej części pracy używane będą następujące terminy:

\begin{itemize}
\item \textit{agent} - program sterujący kontrolowanym obiektem poruszającym się w badanym środowisku,
\item \textit{ekspert} - człowiek lub zewnętrzny program, który zna optymalną (albo przynajmniej skuteczną) politykę działania i potrafi udzielić agentowi informacji na temat realizacji tej polityki i oceniać politykę agenta,
\item \textit{epizod/gra} - ciąg interakcji agenta lub eksperta z otoczeniem, od stanu początkowego do stanu terminalnego.
\item \textit{krok/klatka/krotka/doświadczenie} - pojedyncza informacja $s,a,s',r$ zebrana przez agenta lub eksperta w ramach jednego kroku interakcji ze środowiskiem.  
\item \textit{ocena/odpytanie eksperta} - zarządanie przez program agenta, aby ekspert określił jaką decyzję podjąłby w danym stanie.
\item \textit{trajektoria} - uporządkowana lista krotek pochodząca z jednego epizodu.
\end{itemize}
