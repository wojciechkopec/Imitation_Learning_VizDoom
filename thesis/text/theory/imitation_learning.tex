\section{Uczenie przez Demonstację/Uczenie z Ekspertem}
\subsection{Uczenie przez demonstrację a uczenie nadzorowane}

Najprostszym podejściem do uczenia przez demonstrację jest traktowanie go jak każdego innego problemu uczenia nadzorowanego, przy czym w przeciwieństwie do minimalizowania kosztu działania agenta minimalizowana jest różnica pomiędzy pomiędzy polityką wyuczonego agenta a polityką eksperta. Najprostsze podejście zakłada jednak, że dane uczące i testowe są niezależne i mają jednakowy rozkład, podczas gdy przy uczeniu przez demonstrację nauczona polityka ma bezpośredni wpływ na osiągane później stany, na podstawie których dana polityka będzie sprawdzana. Jak dowiedziono w \cite{DBLP:journals/corr/abs-1011-0686} wynikający z tego błąd rośnie kwadratowo w stosunku do czasu trwania epizodów – gdy klasyfikator popełni błąd w odwzorowywaniu polityki eksperta najprawdopodobniej trafi do stanu nieodwiedzonego przez eksperta, co z dużym prawdopodobieństwem oznacza popełnianie następnych błędów, ponieważ uczeń nie miał jak nauczyć się „podnoszenia się” po błędach.

Jednym ze sposobów radzenia sobie z tym problemem jest wprowadzanie małych zmian podczas iteracji polityki, dzięki czemu rozkład stanów dla nowej polityki jest bliski staremu. Idea polega na zaczynaniu od polityki całkowicie identycznej z polityką eksperta i stopniowym przechodzeniu na politykę wyuczoną. Aby to osiągnąć można wymagać, aby podczas uczenia uczeń mógł w każdej chwili zapytać eksperta, jakie akcje ekspert podjąłby w danym stanie. Dany układ wymaga większej interakcji, ale może być zrealizowany dla wielu z praktycznych przykładów wykorzystania uczenia przez demonstrację.

Pierwszym podejściem opisywanym przez \cite{DBLP:journals/corr/abs-1011-0686} jest uczenie w przód. Podejście opiera się na przeprowadzeniu kilku powtórzeń uczenia, gdzie w każdym kroku następuje uczenie się jednej polityki w jednym, konkretnym, momencie. Jeżeli uczenie będzie przeprowadzone po kolei dla każdego kolejnego kroku w czasie, to próbka uzyskanych stanów, na których prowadzone jest dalsze uczenie odpowiada dystrybucji stanów testowych, a algorytm może odpytać eksperta o właściwe działanie w osiągniętych stanach, dzięki czemu ekspert ma okazję zaprezentować jak „podnosić się” po popełnieniu błędów przez klasyfikator. Powyższe podejście działa tylko dla zadań o skończonym horyzoncie czasowym, wymaga dużej interakcji z ekspertem i możliwości zrestartowania systemu i dokładnego odtworzenia uzyskanego wcześniej stanu, co w wielu przypadkach nie będzie możliwe do zrealizowania.

W celu wyeliminowania tych ograniczeń \cite{DBLP:journals/corr/abs-1011-0686} proponują Iterowany Probabilistyczny Mieszający algorytm. Opierając się na algorytmie iterowania polityki algorytm w każdym kroku stosuje nową stochastyczną politykę wybierając z zadanym prawdopodobieństwem pomiędzy wykonywaniem polityki wyuczonej w poprzednim kroku i konstruowanej w danej iteracji nowej polityki, przy czym prawdopodobieństwo wyboru nowej polityki jest niewielkie. Algorytm zaczyna od dokładnego wykonywania akcji eksperta. W każdej kolejnej iteracji algorytmu prawdopodobieństwo odpytania eksperta jest coraz niższe i zbiega się do 0. Opisane rozwiązanie zostało z powodzeniem przetestowane na przykładzie grania w proste gry, gdzie danymi wejściowymi był obraz z ekranu. Autorzy zdecydowali się na klasyfikator wybierający konkretne akcje dla danego stanu, zamiast częściej używanego w uczeniu ze wzmocnieniem klasyfikatora odwzorowującego funkcję kosztu. Wadą tego podejścia jest brak odrzucania nieskutecznych polityk podczas iteracji, co może prowadzić do niestabilnych wyników.

Wykorzystanie analogicznego rozwiązania proponują \cite{DBLP:journals/corr/BengioVJS15}. Ich propozycja zakłada wybieranie z prawdopodobieństwem $e$ polityki eksperta i z prawdopodobieństwem $1-e$ polityki wyuczonej. Początkowa wartość $e$ powinna wynosić 1, aby klasyfikator mógł nauczyć się  odtwarzać politykę eksperta. Wraz z postępem nauki $e$ powinno stopniowo maleć do 0, aby klasyfikator miał szanse nauczyć się stanów nieodwiedzonych przez eksperta.

W kolejnej publikacji \cite{DBLP:journals/corr/abs-1011-0686} prezentują nowe podejście, nazwane Agregacją Zbioru Danych. W uproszczeniu, podejście to jest następujące: W pierwszej iteracji algorytm zbiera dane testowe stosując politykę pokazaną przez eksperta, po czym trenuje klasyfikator odwzorowujący zachowanie eksperta na danym zbiorze danych. W każdej kolejne iteracji algorytm stosuje politykę wygenerowaną w poprzedniej iteracji i dodaje dane uzyskane podczas jej stosowania do zbioru danych, po czym trenuje klasyfikator by odwzorowywał zachowanie eksperta na całym zbiorze danych. Podobnie jak w poprzednim algorytmie, żeby przyspieszyć uczenie na pierwszych etapach algorytmu, dodano opcjonalną możliwość odpytania eksperta o jego wybór akcji. Uzyskane z pomocą tej metody wyniki są wyraźnie lepsze od wyników uzyskanych za pomocą metody opisanej w poprzednim paragrafie.

\subsection{Podążanie za ekspertem a przewyższanie eksperta}
Dla wielu praktycznych problemów polityka eksperta może nie być optymalna. Algorytm, który stara się tylko i wyłącznie odwzorować politykę eksperta będzie generował w takiej sytuacji nieoptymalne wyniki, które w wielu praktycznych sytuacjach mogą znacznie odbiegać od optimum. Prostym rozwiązaniem tego problemu przedstawionym w \cite{DBLP:journals/corr/ChangKADL15} jest stosowanie e-zachłannej strategii – w każdym ruchu algorytm może wybrać z małym prawdopodobieństwem $e$ wykonanie losowej akcji zamiast akcji optymalnej według wyuczonej polityki. Dzięki temu algorytm może znaleźć lokalne optimum bliskie polityce eksperta. Warto zauważyć, że wymusza to posługiwanie się całościową nagrodą (kosztem) wykonania zadania jako celem optymalizacji, w przeciwieństwie do prostszego minimalizowania różnicy pomiędzy wynikami wyuczonej polityki a polityki eksperta.
