\section{Niepewność - uproszczony eksperyment}
W celu porównania skuteczności Bootstrapa i Dropoutu w ocenianiu niepewności wyników sieci przeprowadzono prostszy eksperyment na lepiej znanych i kontrolowanych danych. Do eksperymentu wykorzystano zbiór MNIST [REF] zawierający odręcznie pisane cyfry. MNIST jest często używany jako przykładowy zbiór danych, służący do przystępnej prezentacji i porównań metod uczenia maszynowego. Najczęściej wykorzystywany jest w kontekscie klasyfikacji, jednak traktowanie wartości liczbowych cyfr jako etykiet (w przeciwieństwie do stosowanego w klasyfikacji one-hot encoding [REF]) w oczywisty sposób odpowiada problemowi regresji.    

\subsection{Eksperyment}
W ramach eksperymentów uczono sieć neuronową na niezbalansowanych zbiorach danych bazujących na MNIST. Dobrane dystrybucje przykładów o konkretnych etykietach w danych uczących mają pokazać, jak poszczególne techniki zachowują się wobec zupełnie nieznanych danych i mało znanych danych.

Zbiór danych MNIST składa się z 60 tysięcy przykładów uczących i 10 tysięcy przykładów treningowych. Każdy obrazek przedstawia jedną czarno-białą cyfrę i ma rozmiar 28x28 pikseli. W eksperymencie etykiety zostały przetrasformowane liniowo z przedziału $[0,9]$ do $[-0.5,0.5]$.

Pierwszy, prostszy, eksperyment posłużył również do znalezienia optymalnych parametrów obu metod. Drugi eksperyment został przeprowadzony z wykorzystaniem znalezionych wcześniej parametrów. Każdy z eksperymentów został powtórzony 10 razy.

\subsubsection{Rozpoznawanie nieznanych danych}
Zachowanie wobec nieznanych danych zbadano ucząc sięć neuronową wyłącznie na przykładach parzystych cyfr. W wynikowej klasyfikacji stopień niepewności zwracanych wyników powinien być znacznie wyższy dla cyfr nieparzystych niż parzystych.


\subsubsection{Ocena stopnia pewności wyników}
Zachowanie wobec mało znanych danych zbadano ucząc sieć neuronową na niezbalansowanym zbiorze danych. Przykłady trafiały do zbioru uczącego z prawdopodobieństwem proporcjonalnym do przedstawianej cyfry, przykładowo 0 z prawdopodobieństwem $p=0$, 5 z prawdopodobieństwem $p=0.5$ i 9 z prawdopodobieństwem $p=0.9$. W wynikowej klasyfikacji stopień niepewności zwracanych wyników powinien być zależny od prawdopodobieństwa trafienia do zbioru danej etykiety.

\subsection{Implementacja bazowa}
Bazą dla implementacji użytych w eksperymencie był przykład klasyfikacji zbioru MNIST za pomocą konwolucyjnych sieci neuronowych udostępniony z biblioteką Tensorflow. Użyta w przykłądzie sieć składa się z dwóch warstw konwolucyjnych i dwóch warstw w pełni połączonych.

W ramach obu eksperymentów uczenie trwa 1 milion iteracji, a wielkość batcha wynosi 128. Prędkość uczenia wynosi $0.005$, przy czym dla Bootstrapu ta wartość jest normalizowana przez średnią liczbę użyć każdego przykładu.

\subsection{Miary jakości}
Kluczowe dla określenia jakości obu metod jest zdefiniowanie miary jakości wyników. Docelowo przyjęta miara powinna dobrze oddawać przydatność do oceny stanu przez agenta DQN.

Za miarę niepewności przyjęto rozstęp międzykwartylowy próbek uzyskanych z sieci. O wyborze rozstępu międzykwartylowego zdecydowała większa od odchylenia standardowego odporność na skrajne wartości. Oprócz rozstępu międzykwartylowego sprawdzono eksperymentalnie również wariancję (która dawała nieznacznie gorsze wyniki) i różnicę między skrajnymi wartościami (zależność od skrajnych wartości uczyniła tą miarę bardzo niestabilną i mało skuteczną).

\[ unc = |q(75) -q(25)|\]

\subsubsection{Rozpoznawanie nieznanych danych}
Liczba znanych i nieznanych etykiet w zbiorze tekstowym jest równa, dlatego za miarę jakości rozdziału danych znanych i nieznanych przyjęto stosunek sumy średnich niepewności dla kolejnych nieznanych etykiet do sumy niepewności dla kolejnych znanych etykiet. Wartości bliskie 1 oznaczają brak rozdziału danych. W eskperymentach wynikowe miary cząstkowe nie przekraczały wartości 2.
\[ quality_{ND} = \frac{\sum_{l \in \{unknown\}} \overline{unc_{l}}}{\sum_{l \in \{known\}} \overline{unc_{l}}}\]

\subsubsection{Ocena stopnia pewności wyników }
Niepewność powinna być odwrotnie proporcjonalna do trafności klasyfikacji, dlatego za miarę jakości oceny niepewności wyników przyjęto wartość absolutną współczynnika korelacji pomiędzy średnią niepewnością a średnią trafnością klasyfikacji dla kolejnych etykiet. Użyty w eksperymencie współczynnik regresji jest liczony dla etykiet od 1 do 9, ponieważ zerowa trafność dla etykiety 0 jest wspólna dla obu metod i zaburza wyraźnie liniową zależność dla reszty etykiet.

\[ quality_{OP} = |r_{unc\ acc}|\]

\subsection{Dropout - konfiguracja}
Dropout został dodany pomiędzy ostatnią warstwą konwolucyjną sieci a pierwszą w pełni połączoną oraz pomiędzy obiema w pełni połączonymi warstwami.
Parametry modelu to prawdopodobieństwo zachowania neuronu w czasie treningu $p_{train}$, prawdopodobieństwo zachowania neuronu w czasie testu $p_{test}$ i liczbę odpytań sieci $n$ przy określaniu niepewności. W eksperymentach sprawdzano wartości $p_{train} \in \{0.25, 0.5, 0.75, 1\}$, $p_{test} \in \{0.25, 0.5, 0.75\}$ i $n \in \{10, 30, 50, 100\}$.

Zastosowana \href{https://link.do.pliku}{implementacja z wykorzystaniem Tensorflow} wymaga każdorazowego przeliczenia wszystkich wartości przy każdym pojedyńczym odpytaniu.

\subsection{Bootstrap - konfiguracja}
Bootstrapowana sieć ma wspólne warstwy konwolucyjne. Z warstw konwolucyjnych wychodzi $n$ niezależnych "głów", skłądających się z dwóch warstwy w pełni połączonych wykorzystanych w przykładzie bazowym. Parametry modelu to liczba "głów" $n$ i prawdopodobieństwo uwzględnienia krotki danych przez głowę $p_{incl}$.  W eksperymentach sprawdzano wartości $n \in  \{5,7,10\}$ i $p_{incl}\in\{0.25, 0.5, 0.75, 1\}$.

Zastosowana \href{https://link.do.pliku}{implementacja z wykorzystaniem Tensorflow} decyduje o uwzględnianiu przez poszczególne głowy dla pełnych batchy danych, a nie dla pojedynczych krotek. Przy odpytywaniu kilku głów o ten sam przykład przeliczenie warstw konwolucyjnych następuje tylko jednokrotnie.


