\section{Wykorzystany agent - implementacja}

Implementacja agenta, który posłużył do wykonania eksperymentów opsiywanych w tym rozdziale znajduje się w pliku \textit{actions\_estimator\_tf.py}. Agent w zależności od ustawień może realizować zarówno algorytmy \textit{kopiowania zachowań} jak i \textit{agregacji zbioru danych}. Implementacja agenta opiera się na głębokiej sieci konwolucyjnej modelującą funkcję wyboru $stan \rightarrow akcja$.


\subsection{Sieć neuronowa}

\subsubsection{Architektura}
Zastosowana architektura opiera się na tej zaproponowanej w pracy \cite{mnih2015human} i wykorzystanej dalej w przykładach środowiska VizDoom i w pracy \cite{DBLP:journals/corr/KempkaWRTJ16}. Sieć składa się kolejno z:

\begin{enumerate}
\item{warstwy konwolucyjnej z 8 filtrami o wielkości 6x6, krokiem 3x3 i wypłaszczoną liniową funkcją aktywacji \textit{(ang. rectified linear unit, ReLU)}}
\item{warstwy konwolucyjnej z 8 filtrami o wielkości 3x3, krokiem 2x2 i wypłaszczoną liniową funkcją aktywacji \textit{(ang. rectified linear unit, ReLU)}}
\item{warstwy dropoutu z prawdopodobieństwem 0.5}
\item{warstwy w pełni połączonej o wielkości 256 neuronów z wypłaszczoną liniową funkcją aktywacji \textit{(ang. rectified linear unit, ReLU)}}
\item{warstwy w pełni połączonej o wielkości 128 neuronów z wypłaszczoną liniową funkcją aktywacji \textit{(ang. rectified linear unit, ReLU)}}
\item{warstwy dropoutu z prawdopodobieństwem 0.5}
\item{warstwy wyjściowej w pełni połączonej z wypłaszczoną liniową funkcją aktywacji \textit{(ang. rectified linear unit, ReLU)}, w której każdy neuron wyjściowy odpowiada jednej akcji}
\item{warstwy softmax}
\end{enumerate}

W stosunku do wersji z pracy pracy \cite{DBLP:journals/corr/KempkaWRTJ16} zastosowana sieć jest rozszerzona o jedną dodatkową warstwę w pełni połączoną i warstwy dropout, ale wszystkie użyte warstwy są złożone z kilka razy mniejszej liczby neuronów.

Liczba i wielkości warstw zostały dobrane eksperymentalnie. Pozostałe parametry sieci, w szczególności parametry filtrów warstw konwolucyjnych zostały wybrane na podstawie wzorców z literatury.

\subsubsection{Pamięć powtórek}
Dla części scenariuszy (\textit{Obrona środka}) problemem przy uczeniu jest niezbalansowanie zbioru danych (niektóre akcje występują znacznie rzadziej niż inne). Zamiast popularnej metody \textit{oversamplingu} zastosowano rozwiązanie, w którym dla każdej z akcji przetrzymywana jest osobna pamięć powtórek. Przy wielkości batcha n i liczbie dostępnych akcji a z każdej z kolejnych pamięci powtórek 
pobierana jest losowa próbka o wielkości n/a. Uzyskane podpróbki łączone są w losowej kolejności i zwracane jako ostateczna próbka.

Implementacja pamięci znajduje się w pliku \textit{replay\_memory.py}.

\subsubsection{Pozostałe ustawienia}

Wagi sieci aktualizowane są za pomocą metody RMSProp \textit{(ang. Root Mean Square Propagation)}. Prędkość uczenia \textit{(ang. learning rate)} została eksperymentalnie ustalona na 0.00025. Sieć jest uczona za pomocą paczek danych \textit{(ang. batch)} o wielkości 64.










